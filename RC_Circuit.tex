\documentclass[12pt]{article}
\usepackage[pagebackref,letterpaper=true,colorlinks=true,pdfpagemode=none,urlcolor=blue,linkcolor=blue,citecolor=blue,pdfstartview=FitH]{hyperref}

\usepackage{amsmath,amsfonts}
\usepackage{graphicx}
\usepackage{color}

\setlength{\oddsidemargin}{0pt}
\setlength{\evensidemargin}{0pt}
\setlength{\textwidth}{6.0in}
\setlength{\topmargin}{0in}
\setlength{\textheight}{8.5in}

\setlength{\parindent}{0in}
\setlength{\parskip}{5px}

%%%%%%%%% For wordpress conversion

\def\more{}

\newif\ifblog
\newif\iftex
\blogfalse
\textrue


\usepackage{ulem}
\def\em{\it}
\def\emph#1{\textit{#1}}

\def\image#1#2#3{\begin{center}\includegraphics[#1pt]{#3}\end{center}}

\let\hrefnosnap=\href

\newenvironment{btabular}[1]{\begin{tabular} {#1}}{\end{tabular}}

\newenvironment{red}{\color{red}}{}
\newenvironment{green}{\color{green}}{}
\newenvironment{blue}{\color{blue}}{}

%%%%%%%%% Typesetting shortcuts

\def\B{\{0,1\}}
\def\xor{\oplus}

\def\P{{\mathbb P}}
\def\E{{\mathbb E}}
\def\var{{\bf Var}}

\def\N{{\mathbb N}}
\def\Z{{\mathbb Z}}
\def\R{{\mathbb R}}
\def\C{{\mathbb C}}
\def\Q{{\mathbb Q}}
\def\eps{{\epsilon}}

\def\bz{{\bf z}}

\def\true{{\tt true}}
\def\false{{\tt false}}

%%%%%%%%% Theorems and proofs

\newtheorem{exercise}{Exercise}
\newtheorem{theorem}{Theorem}
\newtheorem{lemma}[theorem]{Lemma}
\newtheorem{definition}[theorem]{Definition}
\newtheorem{corollary}[theorem]{Corollary}
\newtheorem{proposition}[theorem]{Proposition}
\newtheorem{example}{Example}
\newtheorem{remark}[theorem]{Remark}
\newenvironment{proof}{\noindent {\sc Proof:}}{$\Box$ \medskip} 


\begin{document} 

What happens to the voltage across $R_{o}$ (which we'll call $V_{out}$) if, while the circuit is in a steady-state with $V_{m} = V_{1}$, the input voltage $V_{m}$ switches to $V_{2}$?

\image{width = 300}{}{Images/RC_Circuit.png}

\begin{equation} \label{eq:1} 
i_{m} = i_{c} + i_{o} 
\end{equation}

\begin{equation} \label{eq:2} 
i_{c} = C\frac{dV_{out}}{dt}
\end{equation}

\begin{equation} \label{eq:3} 
i_{o} = \frac{V_{out}}{R_{o}}
\end{equation}

Combining (\ref{eq:1}), (\ref{eq:2}), and (\ref{eq:3}),

\begin{equation} \label{eq:4} 
i_{m} = C\frac{dV_{out}}{dt} + \frac{V_{out}}{R_{o}}
\end{equation}

We also know that

\begin{equation} \label{eq:5} 
i_{m} = \frac{V_{m}-V_{out}}{R_{m}} 
\end{equation}

Combinging (\ref{eq:4}) and (\ref{eq:5}),

\begin{eqnarray*} \label{eq:6} 
0 = C\frac{dV_{out}}{dt} + \frac{V_{out}}{R_{o}} + \frac{V_{out}  - V_{m}}{R_{m}}\\
C\frac{dV_{out}}{dt} + \frac{V_{o}(R_{m} + R_{o})}{R_{o}R_{m}} - \frac{V_{m}}{R_{m}}
\end{eqnarray*}

This is of the form

\begin{equation} \label{eq:7}
0 = a_{1}\frac{dx}{dt} + a_{2}x + a_{3}
\end{equation}

which has a general solution of

\begin{equation} \label{eq:8}
x = \frac{-a_{3}}{a_{2}} + be^{-\frac{a_{2}}{a_{1}}t}
\end{equation}

where b is a constant.

So, 

\begin{equation} \label{eq:9}
V_{out} = \frac{V_{m}R_{o}}{R_{o}+R_{m}} + be^{-\frac{R_{o} + R_{m}}{R_{o}R_{m}C}t}
\end{equation}

Now, solve for initial conditions. (We'll call the time that $V_{m}$ switches from $V_{1}$ to $V_{2}$ $t=0$.) Since the circuit was in steady state,

\begin{equation} \label{eq:10}
V_{out}(0) = \frac{V_{1}R_{o}}{R_{o}+R_{m}}
\end{equation}

Setting (\ref{eq:10}) equal to (\ref{eq:9}), setting t = 0, and solving for b yields

\begin{equation} \label{eq:11}
b = \frac{(V_{1} - V_{2})R_{o}}{R_{o} + R_{m}}
\end{equation}

(Remember, anything raised to the power of 0 equals 1.)

$V_{m}$ is out input, which is $V_{2}$.

So, the solution is

\begin{equation} \label{eq:12}
V_{out} = \frac{V_{2}R_{o}}{R_{o}+R_{m}} + \frac{(V_{1} - V_{2})R_{o}}{R_{o} + R_{m}}e^{-\frac{R_{o}R_{m}C}{R_{o} + R_{m}}t}
\end{equation}

This represents an exponential curve from $V_{1}$ to $V_{2}$, as shown in the following figure.

\image{width = 300}{}{Images/steadyState.png}

Now, what if $V_{m}$ switched back to $V_{1}$ before the circuit reached steady state? $V_{o}$ would curve back to $V_{1}$ according to the following equation, where $V_{i}$ is whatever voltage value $V_{o}$ had when the switch was made.

\begin{equation} \label{eq:13}
V_{out} = \frac{V_{1}R_{o}}{R_{o}+R_{m}} + \frac{(V_{i} - V_{1})R_{o}}{R_{o} + R_{m}}e^{-\frac{R_{o}R_{m}C}{R_{o} + R_{m}}t}
\end{equation}

The following simulations illustrate these equations. In the sim shown above, $V_{1}$ switches to $V_{2}$ forever; $V_{out}$ ramps from $V_{1}$ to $V_{2}$ \footnote{Actually, some value less than $V_{2}$, depending on the exact values of $R_{o}$ and $R_{m}$, which I've set to $10\Omega$ and $1k\Omega$, respesctively.} and stays there\footnote{It never actually gets to a steady-state value. If you examine equation (\ref{eq:13}), you see that the second term really never completely goes to zero.... but, when it gets close enough, we say its in steady-state}. 

If we have $V_{m}$ pulse from $V_{1}$ to $V_{2}$, but switch back to $V_{1}$ before the circuit reaches steady-state, something like this happens:

\image{width = 300}{}{Images/halfway.png}

As we make the pulse shorter, even less of the change in $V_{m}$ gets transfered to the output:

\image{width = 300}{}{Images/shortPulse.png}

So, a "spike" in voltage (like one generated by our motor) will barely transfer to the output at all and our circuit (represented by $R_{o}$) is protected!

\image{width = 300}{}{Images/spike.png}

All code can be found at \url{https://gitthub.com/mathAndCats/RC_Circuit.git}

\end{document}
