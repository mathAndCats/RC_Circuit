\documentclass[12pt]{article}
\usepackage[pagebackref,letterpaper=true,colorlinks=true,pdfpagemode=none,urlcolor=blue,linkcolor=blue,citecolor=blue,pdfstartview=FitH]{hyperref}

\usepackage{amsmath,amsfonts}
\usepackage{graphicx}
\usepackage{color}

\setlength{\oddsidemargin}{0pt}
\setlength{\evensidemargin}{0pt}
\setlength{\textwidth}{6.0in}
\setlength{\topmargin}{0in}
\setlength{\textheight}{8.5in}

\setlength{\parindent}{0in}
\setlength{\parskip}{5px}

\input{macrosblog}

\begin{document} 

What happens to $V_{out}$ if, priot to t=0, the circuit is in a steady-state with $V_{m} = V_{1}$ and at t=0, $V_{m}$ switches to $V_{2}$?

\begin{equation} \label{eq:1} 
i_{m} = i_{c} + i_{r} 
\end{equation}

\begin{equation} \label{eq:2} 
i_{c} = C\frac{dV_{out}}{dt}
\end{equation}

\begin{equation} \label{eq:3} 
i_{r} = \frac{V_{out}}{R_{o}}
\end{equation}

Combining (\ref{eq:1}), (\ref{eq:2}), and (\ref{eq:3}),

\begin{equation} \label{eq:4} 
i_{m} = C\frac{dV_{out}}{dt} + \frac{V_{out}}{R_{o}}
\end{equation}

We also know that

\begin{equation} \label{eq:5} 
i_{m} = \frac{V_{m}-V_{out}}{R_{m}} 
\end{equation}

Combinging (\ref{eq:4}) and (\ref{eq:5}),

\begin{eqnarray*} \label{eq:6} 
0 = C\frac{dV_{out}}{dt} + \frac{V_{out}}{R_{o}} + \frac{V_{out}  - V_{m}}{R_{m}}\\
C\frac{dV_{out}}{dt} + \frac{V_{o}(R_{m} + R_{o})}{R_{o}R_{m}} - \frac{V_{m}}{R_{m}}
\end{eqnarray*}

This is of the form

\begin{equation} \label{eq:7}
0 = a_{1}\frac{dx}{dt} + a_{2}x + a_{3}
\end{equation}

which has a general solution of

\begin{equation} \label{eq:8}
x = \frac{-a_{3}}{a_{2}} + be^{-\frac{a_{2}}{a_{1}}t}
\end{equation}

where b is a constant.

So, 

\begin{equation} \label{eq:9}
V_{out} = \frac{V_{m}R_{o}}{R_{o}+R_{m}} + be^{-\frac{R_{o}R_{m}C}{R_{o} + R_{m}}t}
\end{equation}

Now, solve for initial conditions. Since the circuit was in steady state,

\begin{equation} \label{eq:10}
V_{out}(0) = \frac{V_{1}R_{o}}{R_{o}+R_{m}}
\end{equation}

Setting (\ref{eq:10}) equal to (\ref{eq:9}), setting t = 0, and solving for b yields

\begin{equation} \label{eq:11}
b = -\frac{(V_{1} - V_{0})R_{o}}{R_{o} + R_{m}}
\end{equation}

(Remember, anything raised to the power of 0 equals 1.)

So, the solution is

\begin{equation} \label{eq:12}
V_{out} = \frac{V_{2}R_{o}}{R_{o}+R_{m}} - \frac{V_{2} - V_{1}}{R_{o}}{R_{o} + R_{m}}e^{-\frac{R_{o}R_{m}C}{R_{o} + R_{m}}t}
\end{equation}

This represents an exponential curve from $V_{1}$ to $V_{2}$, as shown in the following figure.

Now, what if $V_{m}$ switched back to $V_{1}$ before the circuit reached steady state? $V_{o}$ would curve back to $V_{1}$ according to the following equation, where $V_{i}$ is whatever voltage value $V_{o}$ had when the switch was made.

\begin{equation} \label{eq:13}
V_{out} = \frac{V_{1}R_{o}}{R_{o}+R_{m}} - \frac{(V_{1} - V_{i})R_{o}}{R_{o} + R_{m}}e^{-\frac{R_{o}R_{m}C}{R_{o} + R_{m}}t}
\end{equation}

Here's a simulation illustrating this, with $\tau$ indicating the time that $V_{2}$ is applied. As $\tau$ gets smaller and smaller, the less $V_{2}$ gets transfered to the output $V_{o}$. This makes the circuit a low-pass filter, as in it passes low frequencies (like DC) and filters out high frequencies (like voltage spikes).

\end{document}
